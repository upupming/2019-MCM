% https://www.latex-tutorial.com/tutorials/tables/

\documentclass{article}


% Align the numbers at the decimal point using the siunitx package
\usepackage{siunitx}
\sisetup {
  % Rounds numbers
  round-mode      = places,
  % to 2 places
  round-precision = 2,
}

% To make a row span several cells
\usepackage{multirow}

% booktabs, alternative for \hline
\usepackage{booktabs}

% To display tables on several pages
\usepackage{longtable} 

% To display tables in landscape
\usepackage{rotating}

\begin{document}
  
\begin{table}[h!]
  \begin{center}
    \caption{Your first table.}
    \label{tab:table1}
    % Alignments: 1st column left, 2nd middle and 3rd right, with vertical lines in between
    \begin{tabular}{l|c|r}
      \textbf{Value 1} & \textbf{Value 2} & \textbf{Value 3} \\
      $\alpha$ & $\beta$ & $\gamma$ \\
      \hline % horizontal line
      1 & 1110.1 & a\\
      2 & 10.1   & b \\
      3 & 23.113231 & c\\
    \end{tabular}
  \end{center}
\end{table}

% Now we align the numbers at the decimal point using the siunitx package

\begin{table}[h!]
  \begin{center}
    \caption{S aligned-number table table.}
    \label{tab:table2}
    % Alignments: 1st column left, 2nd align numbers automatically and 3rd right, with vertical lines in between
    \begin{tabular}{l|S|r}
      \textbf{Value 1} & \textbf{Value 2} & \textbf{Value 3} \\
      $\alpha$ & $\beta$ & $\gamma$ \\
      \hline % horizontal line
      1 & 1110.1 & a\\
      2 & 10.1   & b \\
      3 & 23.113231 & c\\
    \end{tabular}
  \end{center}
\end{table}

% & is column separator
% // is row separator


% Now add one more row
\begin{table}[h!]
  \begin{center}
    \caption{Add one more row.}
    \label{tab:table3}
    % Alignments: 1st column left, 2nd align numbers automatically and 3rd right, with vertical lines in between
    \begin{tabular}{l|S|r}
      \textbf{Value 1} & \textbf{Value 2} & \textbf{Value 3} \\
      $\alpha$ & $\beta$ & $\gamma$ \\
      \hline % horizontal line
      1 & 1110.1 & a\\
      2 & 10.1   & b \\
      3 & 23.113231 & c\\
      4 & 25.113231 & d\\ % <--
    \end{tabular}
  \end{center}
\end{table}

% Or add one more column
\begin{table}[h!]
  \begin{center}
    \caption{Add one more column.}
    \label{tab:table4}
    % Alignments: 1st column left, 2nd align numbers automatically and 3rd right, with vertical lines in between
    \begin{tabular}{l|S|r|l} % <--
      \textbf{Value 1} & \textbf{Value 2} & \textbf{Value 3} & \textbf{Value 4}\\ % <--
      $\alpha$ & $\beta$ & $\gamma$ & $\delta$ \\ % <--
      \hline % horizontal line
      1 & 1110.1 & a & e\\ % <--
      2 & 10.1 & b & f\\ % <--
      3 & 23.113231 & c & g\\ % <--
    \end{tabular}
  \end{center}
\end{table}

% Use multirow
% \multirow{NUMBER_OF_ROWS}{WIDTH}{CONTENT}
\begin{table}[h!]
  \begin{center}
    \caption{Multirow table.}
    \label{tab:table5}
    \begin{tabular}{l|S|r}
      \textbf{Value 1} & \textbf{Value 2} & \textbf{Value 3}\\
      $\alpha$ & $\beta$ & $\gamma$ \\
      \hline
      % Combining 2 rows with arbitrary with (*) and content 12
      \multirow{2}{*}{12} & 1110.1 & a\\ 
      % Content of first column omitted.
      & 10.1 & b\\ 
      \hline
      3 & 23.113231 & c\\
      4 & 25.113231 & d\\
    \end{tabular}
  \end{center}
\end{table}

% Use multicolumn
% \multicolumn{NUMBER_OF_ROWS}{ALIGNMENT}{CONTENT}
\begin{table}[h!]
  \begin{center}
    \caption{Multicolumn table.}
    \label{tab:table6}
    \begin{tabular}{l|S|r}
      \textbf{Value 1} & \textbf{Value 2} & \textbf{Value 3}\\
      $\alpha$ & $\beta$ & $\gamma$ \\
      \hline
      % Combining two cells with alignment c| and content 12.
      \multicolumn{2}{c|}{12} & a\\
      \hline
      2 & 10.1 & b\\
      3 & 23.113231 & c\\
      4 & 25.113231 & d\\
    \end{tabular}
  \end{center}
\end{table}

% Now we combine multirow & multicolumn
\begin{table}[h!]
  \begin{center}
    \caption{Multirow and -column table.}
    \label{tab:table7}
    \begin{tabular}{l|S|r}
      \textbf{Value 1} & \textbf{Value 2} & \textbf{Value 3}\\
      $\alpha$ & $\beta$ & $\gamma$ \\
      \hline
      % Multicolumn spanning 2 columns, content multirow spanning two rows
      \multicolumn{2}{c|}{\multirow{2}{*}{1234}} & a\\
      % Multicolumn spanning 2 columns with empty content as placeholder
      \multicolumn{2}{c|}{} & b\\
      \hline
      3 & 23.113231 & c\\
      4 & 25.113231 & d\\
    \end{tabular}
  \end{center}
\end{table}

% booktabs, alternative for \hline
% We can now replace the hlines in our example table with toprule, midrule and bottomrule provided by the booktabs package:
\begin{table}[h!]
  \begin{center}
    \caption{Table using booktabs.}
    \label{tab:table8}
    \begin{tabular}{l|S|r}
      \toprule % <-- Toprule here
      \textbf{Value 1} & \textbf{Value 2} & \textbf{Value 3}\\
      $\alpha$ & $\beta$ & $\gamma$ \\
      \midrule % <-- Midrule here
      1 & 1110.1 & a\\
      2 & 10.1 & b\\
      3 & 23.113231 & c\\
      \bottomrule % <-- Bottomrule here
    \end{tabular}
  \end{center}
\end{table}


% Multipage tables(too many rows)
% The package longtable: make tables span multiple pages.

% \begin{longtable}[POSITION_ON_PAGE]{ALIGNMENT} 
% Replaces \begin{table}, alignment must be specified here (no more tabular)
\begin{longtable}[c]{l|S|r} 
  %%%%%%%%%%%%%%%%%%%%%
  \caption{Multipage table.}
  \label{tab:table9}\\
  \toprule
  \textbf{Value 1} & \textbf{Value 2} & \textbf{Value 3}\\
  $\alpha$ & $\beta$ & $\gamma$ \\
  \midrule
  % This denotes the end of the header, which will be shown on the first page only
  \endfirsthead
  %%%%%%%%%%%%%%%%%%%%%
  \toprule
  \textbf{Value 1} & \textbf{Value 2} & \textbf{Value 3}\\
  $\alpha$ & $\beta$ & $\gamma$ \\
  \midrule
  %%%%%%%%%%%%%%%%%%%%%
  % Everything between \endfirsthead and \endhead will be shown as a header on every page
  \endhead
  1 & 1110.1 & a\\
  2 & 10.1 & b\\
  % ...
  % ... Many rows in between
  % ...
  3 & 23.113231 & c\\
  \bottomrule
\end{longtable}


% Landscape tables(too many columns)
% Just replace the table environment with the sidewaystable environment
\begin{sidewaystable}[h!]
  \begin{center}
  \caption{Landscape table.}
  \label{tab:table10}
  \begin{tabular}{l|S|r}
  	\toprule
  	\textbf{Value 1} & \textbf{Value 2} & \textbf{Value 3}\\
  	$\alpha$ & $\beta$ & $\gamma$ \\
    \midrule
    1 & 1110.1 & a\\
    2 & 10.1 & b\\
    3 & 23.113231 & c\\
    \bottomrule
  \end{tabular}
  \end{center}
\end{sidewaystable}

\end{document}