% https://www.overleaf.com/learn/latex/LaTeX_Graphics_using_TikZ:_A_Tutorial_for_Beginners_(Part_3)%E2%80%94Creating_Flowcharts

% \documentclass{article}

% \usepackage{tikz}
% \usetikzlibrary{shapes.geometric, arrows}

% \begin{document}

% start & stop blocks,矩形
\tikzstyle{startstop} = [rectangle, rounded corners, minimum width=3cm, minimum height=1cm,text centered, draw=black, fill=red!30]
% input & output box,梯形
\tikzstyle{io} = [trapezium, trapezium left angle=70, trapezium right angle=110, minimum width=3cm, minimum height=1cm, text centered, text width=3cm, draw=black, fill=blue!30]
% process blocks,矩形
% decision blocks,菱形
\tikzstyle{process} = [rectangle, minimum width=3cm, minimum height=1cm, text centered, text width=3cm, draw=black, fill=orange!30]
\tikzstyle{decision} = [diamond, minimum width=3cm, minimum height=1cm, text centered, draw=black, fill=green!30]
% style for the arrows
\tikzstyle{arrow} = [thick,->,>=stealth]


\begin{tikzpicture}[node distance=2cm]

% \node (label) [style name] {text}
\node (start) [startstop] {Start};
\node (in1) [io, below of=start] {Input \\ $E = mc^2$};
\node (pro1) [process, below of=in1] {Process 1};
% \node (dec1) [decision, below of=pro1] {Decision 1};
\node (dec1) [decision, below of=pro1, yshift=-0.5cm] {Decision 1};
\node (pro2a) [process, below of=dec1, yshift=-0.5cm] {Process 2a long text text text text text text text text text text};
\node (pro2b) [process, right of=dec1, xshift=2cm] {Process 2b};
\node (out1) [io, below of=pro2a] {Output \\ $e^{i\theta} = \cos\theta + i\sin\theta$};
\node (stop) [startstop, below of=out1] {Stop};

% Arrows
\draw [arrow] (start) -- (in1);
\draw [arrow] (in1) -- (pro1);
\draw [arrow] (pro1) -- (dec1);
\draw [arrow] (dec1) -- node[anchor=east] {yes} (pro2a);
\draw [arrow] (dec1) -- node[anchor=south] {no} (pro2b);
\draw [arrow] (pro2b) |- (pro1);
\draw [arrow] (pro2a) -- (out1);
\draw [arrow] (out1) -- (stop);

\end{tikzpicture}
% \end{document}